\section{Hardware Design}
In comparison to other Micromouse projects, we laid the focus not so much on
the educational aspect for low-level programming, but on the possiblity to try
out more sophisticated methods of mobile robotics.

\subsection{Microcontroller}
Our design uses the Espressif ESP32-S2-WROOM microcontroller module
featuring an ARM processor clocked at \SI{240}{\mega\hertz},
\SI{320}{\kilo\byte} of SRAM, \SI{4}{\mega\byte} of flash memory and Wi-Fi
connectivity \cite{esp32-s2-wroom-datasheet}.

Espressif provides a software framework that implements a Wi-Fi stack and a
convenient interface to the controller's peripherals for the C language.
\subsection{Sensors and Actuators}
The motor driver circuitry is depicted on page \pageref{schematic:hbridge}.
We would have liked to use a H-bridge driver with integrated logic circuitry
to be able to drive each motor with only two logic signals and integrated
flyback diodes, but because of the higher availability of the part, we chose
the L298P.
The motors have an integrated encoder, making it possible to measure the
angular velocity.
The direction of the motor is chosen by the signals \signal{ESP32\_Mx\_y} and by
applying a PWM signal to the enable pins \signal{H\_EN\_x}, the speed of the
motor can be regulated.

Our design uses three Sharp GP2Y0A51SK0F infrared distance measuring units
visible on page \pageref{schematic:ir}.
These can measure ranges of \SIrange{1.5}{15}{\centi\meter}, which is ideal for
our application where the typical distance of the robot to the wall will be
\SI{3}{\centi\meter}.
Because of their supply current of \SI{20}{\milli\ampere} each we added the
MOSFET \partref{Q6} to be able to implement a low power mode when the robot is
not in use or the battery is low.

Our board has a slot for a board containing an Inertial Measurement Unit, an
ICM-20948 Module by Adafruit with the TDK\,InvenSense ICM-20948 chip that
contains a MEMS accelerometer, a gyroscope and a compass with three axes each
\cite{icm20948-datasheet}.

The module is connected to the microcontroller through a SPI bus, with lines
\signal{MISO} (\signal{SDO}) for data from the module to the controller,
\signal{MOSI} (\signal{SDA}) for data from the controller to the module,
\signal{SCL} as clock,
\signal{INT} for possilbe interrupts sent from the module on movement and
\signal{CS} for addressing the module on the bus.
See the schematics on pages \pageref{schematic:controller} and
\pageref{schematic:imu} for reference.

\subsection{Power Supply}
The power supply unit is documented in the schematics on page
\pageref{schematic:power}.
Power is primarily supplied from a male XT60 socket, so that common lithium
polymer batteries used in R/C model making can be used.
We also made an adaptor to be able to power the robot from a 9V battery.

\begin{table}
\centering
\begin{tabu} to 7.5cm {lX[1,r,p]l}
\toprule
Component & \multicolumn{2}{r}{Current in standby mode} \\
\midrule
ESP32-S2 & \SI{20}{\micro\ampere} \\
USB-UART & \SI{100}{\micro\ampere} \\
Range Sensors & \SI{230}{\micro\ampere} \\
H-Bridge & \SI{6}{\milli\ampere} & (\signal{+5V})\\
& \SI{4}{\milli\ampere} & (\signal{VBAT})\\
IMU & \SI{8}{\micro\ampere} \\
Power Supply & \SI{7}{\milli\ampere} & (\signal{+5V}) \\
& \SI{5}{\milli\ampere} & (\signal{VBAT}) \\
\midrule
Total & \SI{13.4}{\milli\ampere} & (\signal{+5V}) \\
& \SI{9}{\milli\ampere} & (\signal{VBAT}) \\
\bottomrule
\end{tabu}
\caption{Supply current in standby mode.}
\label{tab:standby currents}
\end{table}

If the robot is powered through USB only, the \signal{+5V} line is directly
supplied from the USB bus, through the MOSFET \partref{Q7}.
The AP1509 Buck converter is disabled by the comparator \partref{U6B} through
the \signal{VBUS\_EN} line.
When the battery voltage rises above \SI{6.6}{\volt}, \partref{U6} will
disable the USB power supply and enable the Buck converter.
This way, it is ensured that no current flows back through the USB bus voltage
line.

The initial design had the shutdown pin of \partref{IC1} connected directly to
\signal{USB\_VBUS}, but when powering up the robot from battery, power flows
back through \partref{Q7} to the \signal{USB\_VBUS} line, thereby disabling the
buck converter making it impossible to power the robot from USB.
The schematics on p.\,\pageref{schematic:power} depict the fixed circuit
currently used in the real robot.

For the choice of the \SI{3.3}{\volt} regulator, the high current consumption
of the ESP32-S2 when in Wi-Fi transmission mode had to be considered.
While the LP3965 regulator was chosen because of its
availability at the time of design, it was out of stock, so it had to be
replaced.

As the robot is thought to be powered from a lithium polymer battery, it is
important that the battery is protected from undervoltage.
The robot can crudely measure the battery voltage through a resistor devider
visible on the bottom right on p.\,\pageref{schematic:power}.
In case of the battery voltage becoming too low or the robot laying around
powered, but unused, a standby mode can be entered.
To reduce power consumption, the IMU can be set to sleep mode, the enable pins
of the H-bridge can be driven low and the distance sensors can be powered down
using the MOSFET \partref{Q6}.
Finally, the ESP32-S2 can be set to deep sleep mode.
This way, the total current consumption can be reduced to under
\SI{25}{\milli\ampere} (see \autoref{tab:standby currents}).

The current consumption could be lowered further by using a different H-bridge
driver and a different voltage regulator with lower quiescent currents or,
using a different approach, with a self-holding circuit that drives the
shutdow pin of the Buck converter high, thereby unpowering the whole robot.

While the robot was thought to be powered from a two-cell lithium battery
(\SIrange{6.8}{8.4}{\volt}), the Buck converter allows for a wide input range.
The voltage is limited on the low side by the voltage source switching circuit
which deactivates the Buck converter if the battery voltage drops below
\SI{6.6}{\volt}.
On the high side, the maximum input voltage of the AP1509 is \SI{22}{\volt}.
However, as the maximum measurable voltage for the ADC is
\SI{2500}{\milli\volt}, the maximum voltage measurable by the battery voltage
measurement circuit is around \SI{10}{\volt}.
This way, the ADC will be overdriven for higher voltages.
While the ESP32-S2 is tolerant up to input voltages of \SI{3.6}{\volt}, care
should be taken because the behaviour of the ADC on overvoltage is not
documented in the datasheet.
If higher input voltages need to be used, we recommend altering the values of
the voltage divider accordingly.
For example, replacing \partref{R17} by \SI{10}{\kilo\ohm} will give a
measurable voltage range of up to \SI{25}{\volt} which is higher than the Buck
converter can take in.
