% use latexmk -xelatex 

\documentclass{scrartcl}

\usepackage{amssymb}
\usepackage{amsmath}
\usepackage{tikz}

\usepackage{fontspec}
\usepackage[mathbf=sym]{unicode-math}
\setsansfont{TeX Gyre Heros}
\setmainfont{STIX Two Text}
\setmathfont{STIX Two Math}

\usepackage{siunitx}


\title{Algorithmen für Micromouse}
\author{Fuchs, Alexander}
\date{March 2022}

\newcommand{\bool}{\mathbb{B}}
\newcommand{\RR}{\mathbb{R}}
\newcommand{\TT}{\mathrm{T}}
\newcommand{\true}{\texttt{true}}
\newcommand{\dd}{\mathrm{d}}
\newcommand{\mat}[1]{\mathbf{#1}}
\renewcommand{\|}{\vert}
\newcommand{\eye}[1][]{\mathbb{1}_{#1}}

\begin{document}

\section{Modell des IR-Sensors}
Folgendes Modell $f$ passt ganz gut auf die Spannungswerte des IR-Sensors,
abhängig von der Entfernung:
\begin{align*}
	f_1(x, a, x_0, y_1)
		&= \frac{a}{x + x_0} - \frac{a x_0}{2 (x + x_0)^2} + y_1 \\
	f_2(x, b, y_2)
		&= b x^2 + y_2
\end{align*}
Diese Modelle haben die Eigenschaft, dass jeweils bei $x=0$ ihre Ableitung 0
ist.
Das komplette Modell wird dann gebildet durch:
\[
	f(x) = \left\{\begin{array}{rl}
		f_1\left(x - l, a, x_0, y_1\right) & \text{ für } x \geq l, \\
		f_2\left(x - l, b, y_1 + \frac{a}{2 x_0}\right) & \text{ für } x < l.
	\end{array}\right.
\]

Ein Fit auf beispielhafte Daten eines IR-Sensors ergibt:
\[
	\begin{array}{rcl}
		a &=& \SI{5.324E3}{\milli\volt\per\centi\meter}\\
		b &=& \SI{-1.374E3}{\milli\volt\per\square\centi\meter}\\
		x_0 &=& \SI{1.137E0}{\centi\meter} \\
		y_1 &=& \SI{1.951E1}{\milli\volt} \\
		l &=& \SI{1.201E0}{\centi\meter} \\
	\end{array}
\]
Dabei beträgt die Standardabweichung für die Messpunkte auf der linken Hälfte
\SI{135.5}{\milli\volt} und auf der rechten Hälfte \SI{14.67}{\milli\volt}.

Die Distanz als Funktion der Spannung kann im Bereicht $x \geq l$ gebildet
werden durch Umstellen und Auflösen nach $x$:
\[
	x = \frac{1}{2 (f_1 - y_1)} \left(
		a + \sqrt{a^2 - 2 (f_1 - y_1) a x_0} \right) - x_0
\]


\section{SLAM-Algorithmus}

Die Map der Micromouse lässt sich darstellen als Vektor $\vec{m} \in \bool^M$, wobei jeder Eintrag des Vektors angibt ob eine Wand gesetzt ist oder nicht.
Die Position des Roboters wird im Folgenden als $\vec{x} \in \RR^N$ bezeichnet.
sei der $B_t(\vec{x}, \vec{m}) \in [0, 1]$ der Belief zum Zeitpunkt $t$, also die Wahrscheinlichkeitsverteilung, dass sich der Roboter und die Karte zum Zeitpunkt $t$ im Zustand $(\vec{x}, \vec{m})$ befinden.
Es ist sinnvoll, dass der Roboter und die Welt nicht gekoppelt sind:
$B_t(\vec{x}, \vec{m}) = B_t(\vec{x}) B_t(\vec{m})$.

\subsection{Bayes Filter}
Sei $u_t$ der Steuerbefehl zur Zeit $t$, dann lautet der
\textbf{prediction step}:
\begin{align}
    \tilde B_t(\vec{x}, \vec{m}) = \tilde B_t(\vec{x}) \tilde B_t(\vec{m})
    &= \int \int p(\vec{x}, \vec{m} \| u_t, \vec{x}', \vec{m}') B_{t-1}(\vec{x}', \vec{m}') \dd\vec{x}' \dd\vec{m}'
    \\
    &= \int \int p(\vec{x}, \| u_t, \vec{x}', \vec{m}') p(\vec{m} \| \vec{m}')
    B_{t-1}(\vec{x}', \vec{m}') \dd\vec{x}' \dd\vec{m}'
    \\
    &= \int \int p(\vec{x}, \| u_t, \vec{x}', \vec{m}') \delta(\vec{m} - \vec{m}')
    B_{t-1}(\vec{x}', \vec{m}') \dd\vec{x}' \dd\vec{m}'
    \\
    &= \int p(\vec{x}, \| u_t, \vec{x}', \vec{m}) B_{t-1}(\vec{x}', \vec{m}) \dd\vec{x}'
    \\
    &= \int p(\vec{x}, \| u_t, \vec{x}', \vec{m}) B_{t-1}(\vec{x}') B_{t-1}(\vec{m}) \dd\vec{x}'
\intertext{also, da sich die Map zeitlich nicht ändern darf}
    \tilde B_t(\vec{m}) &= B_{t-1}(\vec{m}),\\
    \tilde B_t(\vec{x}) &= \int p(\vec{x}, \| u_t, \vec{x}', \vec{m}) B_{t-1}(\vec{x}') \dd\vec{x}'
\end{align}
Und das \textbf{measurement update}, mit $z_t$ die Messung:
\[
    B_t(\vec{x}, \vec{m}) = \eta p(z_t \| \vec{x}, \vec{m}) \tilde B_t(\vec{x}, \vec{m})
\]
mit dem Normalizer $\eta$.
Daraus folgt durch Integration über $x$ bzw. Summation über Einstellmöglichkeiten von $\vec{m}$:
\begin{align}
    B_t(\vec{x}) &= \sum_{\vec{m}} B_t(\vec{x}) B_t(\vec{m}) = 
    \eta \sum_{\vec{m}} p(z_t \| \vec{x}, \vec{m}) \tilde B_t(\vec{x}) \tilde B_t(\vec{m})
    \\
    B_t(\vec{m}) &= \int B_t(\vec{x}) B_t(\vec{m}) \dd\vec{x} = 
    \eta \int p(z_t \| \vec{x}, \vec{m}) \tilde B_t(\vec{x}) \tilde B_t(\vec{m}) \dd\vec{x}
\end{align}

% In einem Partikelfilter könnte die Summe als Summe über mögliche Zustände der Map für jeden Partikel, das Integral als Summe über die Partikel dargestellt werden.
Das Problem hier ist, dass das Feld mit $8 \times 7 \times 2$ Brettern schon $2^{112}$ mögliche Karten besitzt.
Man kann die Summation vereinfachen, indem man nur Wände in der lokalen Umgebung betrachtet, die auch von den Sensoren erreichbar sind:

\begin{tikzpicture}
    \draw (0, 1) -- (3, 1) -- (3, 2) -- (2, 2) -- (2, 0) -- (1, 0) -- (1, 2) -- (0, 2) -- cycle;
    \draw[red] (1.3, 1.7) -- (1.7, 1.7) -- (1.5, 1.2) -- cycle;
\end{tikzpicture}

Dann hat man nur noch $2^{12} = 4096$ Zustände zu summieren.
Das ist immer noch unglaublich viel.
Man kann noch unter den drei Sensoren die sichtbaren Wände aufteilen:
\[
    B_t(\vec{x}) = \sum_{\vec{m}} B_t(\vec{x}) B_t(\vec{m}) = 
    \eta \tilde B_t(\vec{x}) \sum_{i=1}^3 \sum_{\vec{m}} p(z_{i,t} \| \vec{x}, \vec{m}_i) \tilde B_t(\vec{m}_i)
\]
Dann muss man $3 \times 2^4 = 48$ Zustände betrachten und das sieht schon viel realistischer aus.
Wie viele Wände mit den IR-Sensoren in Wirklichkeit sichtbar sein könnten, wird sich bei Tests zeigen.

Auch ausprobieren könnte man, einfach mit Wahrscheinlichkeit $\tilde B_t(\vec{m})$ ein paar Maps zu sampeln.

\subsection{Kalman-Filter}
Ein Kalman-Filter wird zwar unter Umständen nicht ausreichen, ich möchte ihn hier aber durchrechnen weil er einfach ist.

Der Zustandsübergang ist hier linear modelliert mit $\vec{x}_t = \mat{A}_t \vec{x}_{t-1} + \mat{B}_t \vec{u}_t + \vec{\epsilon}_t$.
Damit gilt für die prediction:
\begin{align}
    \tilde{\vec{x}}_t &= \mat{A}_t \vec{x}_{t-1} + \mat{B}_t \vec{u}_t,
    \\
    \mat{\tilde \Sigma}_t &= \mat{A}_t \mat{\Sigma}_{t-1} \mat{A}_t^\TT + \mat{R}_t.
\end{align}
Dabei ist $\mat R$ die durch Übergangsrauschen verursachte Kovarianz.

Die Messung hat die Wahrscheinlichkeit $p(\vec{z}_t \| \vec{x}_t, \vec{m}_t)$ sei bekannt, wir brauchen aber 
\[
    p(\vec{z}_t \| \vec{x}_t) = \sum_{\vec{m}} p(\vec{z}_t, \vec{m} \| \vec{x}_t)
	= \sum_{\vec{m}} p(\vec{z}_t \| \vec{x}_t, \vec{m}) B_{t-1}(\vec{m}).
\]
Damit wird die lineare Messung:
\begin{align}
	\vec{z}_t = \sum_{\vec{m}} B_{t-1}(\vec{m}) \mat{C}_t(\vec{m}) \vec{x}_t + \vec{\delta}_t
\end{align}
mit einem Rauschen $\vec{\delta}_t$ mit Kovarianz $\mat{Q_t} = \sum_{\vec m}
B_{t-1}(\vec{m}) \mat{Q}_t(\vec{m})$.
Weiter macht man dann das ganz normale Kalman measurement update:
\begin{align}
    \mat{K}_t &= \mat{\tilde\Sigma}_t \mat{C}_t^\TT (\mat{C}_t \mat{\tilde\Sigma} \mat{C}_t^\TT + \mat{Q}_t)^{-1}
    \\
    \vec{x}_t &= \tilde{\vec{x}}_t + \mat{K}_t (\vec{z}_t - \mat{C}_t \tilde{\vec{x}}_t )
    \\
    \mat{\Sigma}_t &= (\eye - \mat{K}_t \mat{C}_t) \mat{\tilde\Sigma}_t
\end{align}

\subsection{Filter für die Map}
Die Map kann dargestellt werden durch einen Vektor $\vec{\pi} \in [0, 1]^M$,
wobei $\pi_i$ die Wahrscheinlichkeit darstellt, dass $m_i = \true$.



\end{document}
